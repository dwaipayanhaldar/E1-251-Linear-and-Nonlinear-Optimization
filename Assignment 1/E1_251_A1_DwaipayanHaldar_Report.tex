\documentclass[12pt,a4paper,onecolumn]{exam}

\usepackage{amsmath}
\usepackage{amsthm}
\usepackage{amssymb}
\usepackage{graphicx}
\usepackage{float}
\usepackage{geometry}
\usepackage{tikz}
\usepackage[skins]{tcolorbox}
\usepackage{listings}
\usepackage[font=small]{caption}
\usepackage{subcaption}

% --- Code Listing Colors ---
\definecolor{codegreen}{rgb}{0,0.6,0}
\definecolor{codegray}{rgb}{0.5,0.5,0.5}
\definecolor{codepurple}{rgb}{0.58,0,0.82}
\definecolor{backcolour}{rgb}{0.95,0.95,0.92}

% --- Code Listing Style ---
\lstdefinestyle{mystyle}{
    backgroundcolor=\color{backcolour},
    commentstyle=\color{codegreen},
    keywordstyle=\color{magenta},
    numberstyle=\tiny\color{codegray},
    stringstyle=\color{codepurple},
    basicstyle=\ttfamily\footnotesize,
    breaklines=true,
    captionpos=b,
    keepspaces=true,
    numbers=left,
    numbersep=2pt,
    showspaces=false,
    showstringspaces=false,
    showtabs=false,
    tabsize=1
}

\lstset{style=mystyle}

\begin{document}

\renewcommand{\qedsymbol}{$\blacksquare$}
\begingroup
    \centering
    \LARGE \textbf{E1 251 Project: Reconstruction from Non-Uniform Samples Using a DCT-$l_p$ Prior}\\
    \phantom{123} \\
    \large \textbf{Dwaipayan Haldar(SR. No.: 27128)}\par
\endgroup

\noindent\rule{\textwidth}{0.5pt}

\section{Derivation of MM-CG Algorithm}

\[
J(x) = \|Wx - m\|_2^2 + \lambda \sum_{i=1}^N (\varepsilon + (\text{DCT } x)_i^2)^p
\]
Focusing our attention the second term, we can see that $\lambda \sum_{i=1}^N (\varepsilon + (\text{DCT } x)_i^2)^p$ is concave for the given range $0.2 < p < 0.5$. So the cost function can be re-written as 
\[
J(\mathbf{x}) = f_0(\mathbf{x}) + f_{ccv}(\mathbf{x})
\]
where $f_0(x)$ is the convex term $\|W\mathbf{x} - m\|_2^2$ and $f_{ccv}(\mathbf{x}^{(k)})$ is $\lambda \sum_{i=1}^N (\varepsilon + (\text{DCT} \mathbf{x})_i^2)^p$. For the concave term, the linear approximation at $\mathbf{x}^{(k)}$ maximizes the function around $\mathbf{x}^{(k)}$. We can thus write 
\[
f_{ccv}(\mathbf{x}) \leq f_{ccv}(\mathbf{x}^{(k)}) + \nabla f_{ccv}(\mathbf{x}^{(k)})^T(\mathbf{x} - \mathbf{x}^{(k)})\\ 
\]
So, we can write
\[
J(\mathbf{x}) \leq f_0(\mathbf{x}) + \nabla f_{ccv}(\mathbf{x}^{(k)})^T\mathbf{x} + \text{constant}\\
\] 
Let $z_i = (DCT \mathbf{x})_i^2$ and  $g(\mathbf{z}) = \sum_{i=1}^{N}(\varepsilon + z_i)^p$, where $\mathbf{z} = [z_1 z_2 ... z_N]^T$. Then $f_{ccv}(\mathbf{x})= \lambda\mathbf{z}$. Using the above equations we can write,
\[
\begin{aligned}
    g(\mathbf{z}) &\leq \nabla g(\mathbf{z}^{(k)})^T\mathbf{z} + \text{constant} \\
\frac{\partial g(\mathbf{z}^{(k)})}{\partial z_i^{(k)}} = p(\varepsilon + z_i^{(k)})^{p-1} &= p(\varepsilon + (DCT \mathbf{x}^{(k)})_i^2)^{p-1} =: w_i^{(k)}(\text{By Definition}) \\
\implies \nabla g(\mathbf{z}^{(k)}) = \big[w_1^{(k)} w_2^{(k)} \cdots w_N^{(k)}   \big]^T &\implies g(\mathbf{z}) \leq \big[w_1^{(k)} w_2^{(k)} \cdots w_N^{(k)}   \big]^T \odot \mathbf{z} + \text{const} \\ 
\implies g(\mathbf{z}) \leq \big[w_1^{(k)} w_2^{(k)} \cdots w_N^{(k)}   \big]^T \odot (&DCT \mathbf{x})^2 + \text{const} = \sum_{i = i}^{N} w_i^{(k)}(DCT \mathbf{x})_i^2 + \text{const} \\
\end{aligned}
\]
So, the quadratic surrogate of $J(\mathbf{x})$ is:
\[
J(\mathbf{x}) \leq \|W\mathbf{x} - m\|_2^2 + \lambda \sum_{i=1}^N w_i^{(k)} (DCT \mathbf{x})_i^2 + \text{Constant}
\]
$DCT\mathbf{x}$ can be replaced by $C\mathbf{x}$ as it is a linear transform of $\mathbf{x}$. And $IDCT\mathbf{x}$ can be replaced by $C^T\mathbf{x}$. Let $W_k = diag(w_1^{(k)} w_2^{(k)} \cdots w_N^{(k)})$. Doing these replacements, the equation can be rewritten as:
\[
J(\mathbf{x}) \leq \|W\mathbf{x} - m\|_2^2 +  \lambda (C\mathbf{x})^T W_k (C\mathbf{x}) + \text{Constant} = Q(\mathbf{x})(\text{let})
\]
Making the gradient equals to 0 to get the minimum value.
\[
\nabla Q(\mathbf{x}) = (2 W^T W \mathbf{x} - 2 W^T \mathbf{m}) + (2 \lambda C^T W_k C \mathbf{x}) = \mathbf{0}
\] 
The equation can be rewritten as 
\[
(W^T W + \lambda C^T W_k C)(\mathbf{x}) = W^T\mathbf{m}
\]
which is basically the equation as:
\[
(W^T W + \lambda \text{IDCT}(\text{diag}(w^{(k)}) \text{ DCT}))\mathbf{x} = W^T\mathbf{m}
\] \qed

\section{Description of Experimental setup}


\end{document}